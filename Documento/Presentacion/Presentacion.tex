%%%%%%%%%%%%%%%%%%%%%%%%%%%%%%%%%%%%%%%%%
% Beamer Presentation
% LaTeX Template
% Version 1.0 (10/11/12)
%
% This template has been downloaded from:
% http://www.LaTeXTemplates.com
%
% License:
% CC BY-NC-SA 3.0 (http://creativecommons.org/licenses/by-nc-sa/3.0/)
%
%%%%%%%%%%%%%%%%%%%%%%%%%%%%%%%%%%%%%%%%%

%----------------------------------------------------------------------------------------
%	PACKAGES AND THEMES
%----------------------------------------------------------------------------------------

\documentclass[mathserif]{beamer}

\mode<presentation> {

% The Beamer class comes with a number of default slide themes
% which change the colors and layouts of slides. Below this is a list
% of all the themes, uncomment each in turn to see what they look like.

%\usetheme{default}
%\usetheme{AnnArbor}
\usetheme{Antibes} %++
%\usetheme{Bergen}
%\usetheme{Berkeley}
%\usetheme{Berlin}
%\usetheme{Boadilla}
%\usetheme{CambridgeUS} %+
%\usetheme{Copenhagen}
%\usetheme{Darmstadt} %++
%\usetheme{Dresden}
%\usetheme{Frankfurt}
%\usetheme{Goettingen}
%\usetheme{Hannover}
%\usetheme{Ilmenau} %++
%\usetheme{JuanLesPins}
%\usetheme{Luebeck}
%\usetheme{Madrid}
%\usetheme{Malmoe}
%\usetheme{Marburg}
%\usetheme{Montpellier}
%\usetheme{PaloAlto}
%\usetheme{Pittsburgh}
%\usetheme{Rochester}
%\usetheme{Singapore}
%\usetheme{Szeged}
%\usetheme{Warsaw}

% As well as themes, the Beamer class has a number of color themes
% for any slide theme. Uncomment each of these in turn to see how it
% changes the colors of your current slide theme.

%\usecolortheme{albatross}
\usecolortheme{beaver}
%\usecolortheme{beetle}
%\usecolortheme{crane}
%\usecolortheme{dolphin}
%\usecolortheme{dove}
%\usecolortheme{fly}
%\usecolortheme{lily}
%\usecolortheme{orchid}
%\usecolortheme{rose}
%\usecolortheme{seagull}
%\usecolortheme{seahorse}
%\usecolortheme{whale}
%\usecolortheme{wolverine}

%\setbeamertemplate{footline} % To remove the footer line in all slides uncomment this line
\setbeamertemplate{footline}[frame number] % To replace the footer line in all slides with a simple slide count uncomment this line

\setbeamertemplate{navigation symbols}{} % To remove the navigation symbols from the bottom of all slides uncomment this line
}

\usepackage{graphicx} % Allows including images
\usepackage{booktabs} % Allows the use of \toprule, \midrule and \bottomrule in tables
\usepackage[utf8x]{inputenc}
\usepackage[spanish]{babel}
\usepackage{tikz,times}
\usepackage{multicol}
\usepackage{verbatim}
\usetikzlibrary{mindmap,trees,backgrounds}

  % Keys to support piece-wise uncovering of elements in TikZ pictures:
  % \node[visible on=<2->](foo){Foo}
  % \node[visible on=<{2,4}>](bar){Bar}   % put braces around comma expressions
  %
  % Internally works by setting opacity=0 when invisible, which has the 
  % adavantage (compared to \node<2->(foo){Foo} that the node is always there, hence
  % always consumes space plus that coordinate (foo) is always available.
  %
  % The actual command that implements the invisibility can be overriden
  % by altering the style invisible. For instance \tikzsset{invisible/.style={opacity=0.2}}
  % would dim the "invisible" parts. Alternatively, the color might be set to white, if the
  % output driver does not support transparencies (e.g., PS) 
  %
\tikzset{
    invisible/.style={opacity=0},
    visible on/.style={alt={#1{}{invisible}}},
    alt/.code args={<#1>#2#3}{%
      \alt<#1>{\pgfkeysalso{#2}}{\pgfkeysalso{#3}} % \pgfkeysalso doesn't change the path
    },
  }

%----------------------------------------------------------------------------------------
%	TITLE PAGE
%----------------------------------------------------------------------------------------

\title[Evaluación de modelos de aprendizaje automático
para posicionamiento indoor utilizando Bluetooth
low energy]{Evaluación de modelos de aprendizaje automático
para posicionamiento indoor utilizando Bluetooth
low energy\\\normalsize Trabajo de Memoria} % The short title appears at the bottom of every slide, the full title is only on the title page

\author{Felipe Berrios Toloza} % Your name
\institute[UTFSM] % Your institution as it will appear on the bottom of every slide, may be shorthand to save space
{
Universidad Técnica Federico Santa María \\ % Your institution for the title page
\medskip
\textit{felipe.berriost@alumnos.usm.cl} % Your email address
}
\date{11 de abril de 2018} % Date, can be changed to a custom date

\begin{document}

\begin{frame}
\titlepage % Print the title page as the first slide
\end{frame}

\begin{frame}{Tabla de Contenidos}
\begin{multicols}{2}
  \tableofcontents
\end{multicols}
\end{frame}


\AtBeginSection[]
{
  \begin{frame}{Tabla de Contenidos}
   \begin{multicols}{2}
     \tableofcontents[currentsection,hideothersubsections]
   \end{multicols}
  \end{frame}
}

%----------------------------------------------------------------------------------------
%	PRESENTATION SLIDES
%----------------------------------------------------------------------------------------

%------------------------------------------------
\section{Introducción} 
%------------------------------------------------

\begin{frame}
\frametitle{Introducción}

\textbf{Geolocalización}
\begin{itemize}

%\item La geolocalización ha jugado un papel fundamental en las últimas décadas.

\item Desde la edad antigua, múltiples formas de localización han sido desarrolladas.

\item Dentro de los avances más importantes en este ámbito, es el desarrollo de la teoría científica y técnica denominada georreferenciación.

\item Gracias a GPS, el crecimiento y acceso de la georreferenciación y navegación está en progresivo aumento.

\item Motivación: Georreferenciar dentro de una explotación minera, donde no hay alcance de señales GPS.

\end{itemize}



\end{frame}


%------------------------------------------------
\subsection{Definición del problema}

\begin{frame}
\frametitle{Definición del problema}

\begin{itemize}

\item Es necesario posicionamiento en interiores (Mall, Evacuación, Mineras, edificios subterráneos)

\pause
\item Cuando se usa tecnología GPS dentro de edificios o bajo tierra, existen muchos obstáculos e interferencia que imposibilitan su uso.

\pause
\item Sistemas de posicionamiento actuales (IPS) presentan problemas ya que confían en indicadores que son afectados por ruido como el indicador de fuerza de la señal (RSSI).

\end{itemize}

\pause
\vspace*{.5cm}
\textbf{Problema: Mejorar exactitud de sistemas de posicionamiento en interiores mediante modelos que aprendan de las señales}

\end{frame}

%------------------------------------------------

\subsection{Objetivos} % A subsection can be created just before a set of slides with a common theme to further break down your presentation into chunks

\begin{frame}
\frametitle{Objetivos}

\begin{itemize}
%\item Evaluar calidad de las señales \textit{Bluetooth Low Energy}, tanto en precisión como exactitud.

\item Diseñar un método de mapeo para un área mediante señales RSSI (\textit{fingerprint}).
\pause
\item Comparar métodos de aprendizaje automático sobre mediciones RSSI para determinar cúal posee menor error y es más exacto.
\pause
\item Determinar que tanto afectan los métodos de reducción de dimensionalidad tanto en precisión, error y tiempo de procesamiento para los algoritmos de máquinas de aprendizaje estudiados.

\end{itemize}

\end{frame}

%------------------------------------------------
\section{Estado del Arte}
%------------------------------------------------

\subsection{Tecnologías para posicionamiento \textit{indoor}}
\begin{frame}
\frametitle{Tecnologías para posicionamiento \textit{indoor}}

 % The "c" option specifies centered vertical alignment while the "t" option is used for top vertical alignment

\only<1>{Basado en Visión

\begin{figure}
\includegraphics[width=\textwidth]{../figures/Murallas.png}
\end{figure}}

\only<2>{Infrarrojo

\begin{itemize}

\item Transmisor infrarrojo con un identificador único.

\item Receptores son colocados en lugares dentro del recinto, los cuales pueden detectar este identificador único y comunicar a un software especializado.

\item No se afecta por interferencia electromagnética. Costoso y complejo.
\end{itemize}
}

\only<3>{Tecnologías basadas en Sonido

\begin{figure}
\includegraphics[width=0.8\textwidth]{../figures/abs.png}
\end{figure}
}

\only<4>{RFID

\begin{itemize}
\item La localización mediante RFID puede categorizarse en dos tipos, los cuales son localización del lector y localización de tags.

\item Costoso y no escalable.

\item Poco alcance, sin embargo no necesita linea de visión directa.
\end{itemize}
}

\only<5>{Tecnologías Inalámbricas

\begin{block}{Received Signal Strength Indicator}
RSSI es una escala de referencia para medir el nivel de potencia de la fuerza de la señal recibida por el receptor. Se mide en dBm donde 0 RSSI indica señal ideal y valores más negativos indican mayor perdida. 
\end{block}

\begin{block}{Tx Power}
Es la potencia de salida o fuerza de la señal que el emisor produce durante el tiempo de transmisión. A mayor Tx Power, más estable es la señal, pero más energía se consume.
\end{block}
}

\only<6> { Comparativa de tecnologías

\begin{figure}
\includegraphics[width=0.8\textwidth]{../figures/comparativa.png}
\end{figure}

}


\end{frame}

%------------------------------------------------
\subsection{Técnicas  matemáticas Wireless para localización indoor}
\begin{frame}
\frametitle{Proximidad}

\begin{itemize}

\item Es el método más simple, y se basa en determinar una posición simbólica y aproximada de la posición del usuario.

\item Antenas o emisores de ondas de radio. Según la señal más fuerte detectada por el usuario, es donde se localiza en el sistema.

\item Ampliamente usado en redes celulares, ya que permite determinar la posición de un dispositivo con una precisión de 50-200 m, sin embargo, no es buena en espacios reducidos. GSM, Infrarrojo, Cell-ID.
\end{itemize}


\end{frame}

%------------------------------------------------

\begin{frame}
\frametitle{Triangulación}

\begin{figure}
\includegraphics[width=\linewidth]{../figures/triangulacion.png}
\end{figure}

\end{frame}

%------------------------------------------------

\begin{frame}
\frametitle{Fingerprint}

\begin{figure}
\includegraphics[width=.8\linewidth]{../figures/finger.png}
\end{figure}

\end{frame}


%------------------------------------------------
\section{Propuesta de solución}
%------------------------------------------------

\begin{frame}
\frametitle{Propuesta}

\begin{itemize}
\item Establecer un marco de trabajo para la recolección, entrenamiento y clasificación de algoritmos de machine learning utilizando Bluetooth Low Energy.\\

\pause

\item Comparación de diferentes clasificadores.\\

\pause
\item Utilizar técnicas de reducción de dimensionalidad.\\
\pause

\item Utilizar modelos sin necesidad de conexión a internet.
\end{itemize}

\end{frame}


%------------------------------------------------
\subsection{Consideraciones Previas}

\begin{frame}
\frametitle{Beacons}

\begin{columns}[t] % The "c" option specifies centered vertical alignment while the "t" option is used for top vertical alignment

\column{.5\textwidth} % Left column and width

\begin{itemize}
\item \onslide<1->{La transmisión corresponde a un ID único que está presente en cada Beacon y que no se repite, como una dirección MAC o un UUID.}

\item \onslide<2->{Auge del Internet de las cosas.}

\item \onslide<3->{Habitualmente los Beacons soportan ambos protocolos existentes, es decir IBeacon y Eddystone.}

\end{itemize}

\column{.5\textwidth} % Right column and width
\begin{figure}
\includegraphics[width=\textwidth]{../figures/beacons_all.jpg}

\end{figure}

\end{columns}

\end{frame}

%------------------------------------------------
\begin{frame}
\frametitle{Beacons - Valores esperados}

\begin{table}[ht!]
\centering
\resizebox{\columnwidth}{!}{
\begin{tabular}{|c|c|c|}
Parámetro                   & Kontakt.io                      & Estimote                             \\ \hline
Duración de la batería      & Hasta 4 años                    & Hasta 2 años                         \\ \hline
Rango                       & 70m                             & 70m                                  \\ \hline
Procesador                  & 32-bit ARM® Cortex™ M0 CPU core & ARM® Cortex®-M4 32-bit processor FPU \\ \hline
Sensibilidad                & -93dBm                          & -96 dBm                              \\ \hline
Velocidades                 & 250kBs, 1Mbs, y 2Mbs            & 1 Mbps (2 Mbps soportado)            \\ \hline
Memoria                     & 256KB flash 16KB RAM            & 512 kB Flash memory 64 kB RAM memory \\ \hline
Transmission power          & -30dBm a 4dBm                  & -20dBm a +4 dBm                        \\ \hline
Batería                     & 2 x 1.000mAh CR2477             & 1 x CR2477 – 3.0V                    \\ \hline
Bluetooth                   & Bluetooth® 4.2 LE standard      & Bluetooth® 4.2 LE standard           \\ \hline
Espesor                     & 15mm                            & 17mm                                 \\ \hline
Peso                        & 35 gr                           & 30 gr                                \\ \hline
Paquete IBeacon y Eddystone & 1 a la vez                      & 1 a la vez                           \\ \hline
Paquetes adicionales        & telemetría                      & telemetría                           \\ \hline
Sensores adicionales        & Temperatura                     & movimiento, temperatura              \\ \hline
Batería reemplazable        & Si                              & Si                                   \\ \hline
Numero de Beacons           & 3                               & 3                                    \\ \hline
Precio                      & 60 USD                          & 59 USD                               \\ \hline
\end{tabular}
}
\end{table}
\end{frame}

%------------------------------------------------

\begin{frame}
\frametitle{Estabilidad de la señal Bluetooth}

\begin{itemize}
\item Se realiza prueba para comprobar como afecta las interferencias a la señal Bluetooth.

\end{itemize}

\begin{figure}
\includegraphics[width=.7\textheight]{../figures/mediciones_beacon_interferencia.jpg}
\end{figure}

\end{frame}

%------------------------------------------------

\begin{frame}
\frametitle{Algoritmos de \textit{Machine Learning}}

\begin{columns}[t] % The "c" option specifies centered vertical alignment while the "t" option is used for top vertical alignment

\column{.33\textwidth}

\begin{center}k-NN\end{center}

\begin{figure}
\includegraphics[width=\textwidth]{../figures/knn.png}
\end{figure}

\pause

\column{.33\textwidth}

\begin{center}SVM\end{center}

\begin{figure}
\includegraphics[width=\textwidth]{../figures/SVM-Planes.png}
\end{figure}

\pause

\column{.33\textwidth}

\begin{center}Neural Networks\end{center}

\begin{figure}
\includegraphics[width=\textwidth]{../figures/deep.png}
\end{figure}

\end{columns}

\end{frame}

%------------------------------------------------
\subsection{Descripción del \textit{framework} de posicionamiento}
\begin{frame}
\frametitle{Descripción del \textit{framework} de posicionamiento}

\begin{itemize}

\item Establecer un marco de trabajo.

\pause

\item Se utiliza la técnica de Fingerprint discutida en el estado del arte, mediante la utilización de un mapa de señales, también denominado \textit{radiomap}.

\pause

\item Utilizar dispositivos Bluetooth Low Energy, lo cuales realizan la función de access point(AP) y que serán los responsables de emitir la señal RSSI. Luego, el procedimiento se divide en las dos clásicas etapas de Fingerprint, es decir, fase \textit{offline} y fase \textit{online}.
\end{itemize}


\end{frame}

%------------------------------------------------


\begin{frame}
\frametitle{Fase Offline}

\begin{itemize}
\item Crear un tipo de aplicación que sea capaz de recolectar los vectores RSSI.

\pause

\item El periodo y frecuencia de los datos se debe determinar experimentalmente. Para ello, cada medición a colectar representa un punto en el espacio $2-dimensional$.

\pause

\item Para generar la grilla, es necesario tener la posición exacta, que corresponde a la etiqueta de cada punto.

\pause

\item Con los datos registrados, se debe crear la base de datos que almacenara estos Fingerprints, ya que desde ahí es posible analizar los datos y mantener su persistencia. Posteriormente, con estos datos se crea el radiomap.
\end{itemize}


\end{frame}

%------------------------------------------------

\begin{frame}
\frametitle{Fase Offline}

\begin{figure}
\includegraphics[width=\textwidth]{../figures/fingerprints.jpg}
\end{figure}

\end{frame}

%------------------------------------------------

\begin{frame}
\frametitle{Reducción de dimensionalidad}

\begin{itemize}

\item Esto no ha sido mayormente explorado en la literatura

\pause
\item Existe correlación espacial lineal de las señales adyacentes. PCA ayuda a eliminar esta correlación.
\pause

\item Los métodos de extracción de características pueden ayudar a agilizar la fase de entrenamiento, ya que este proceso es lento. Además, al ser menos componentes, en la fase online, las técnicas tardaran mucho menos tiempo en determinar la posición en tiempo real.

\pause

\item Descubrir atributos en un espacio no correlacionado. Transformación lineal del vector RSSI.

\end{itemize}

\end{frame}

%-------------------------------------------------

\begin{frame}
\frametitle{Entrenamiento de algoritmos}

\begin{itemize}

\item Entrenar técnicas de máquinas de aprendizaje muy conocidos y que han presentado buenos resultados a lo largo de muchos problemas.

\pause
\item Posteriormente, se seleccionan los mejores algoritmos, es decir, que presenten el mejor desempeño y luego son implementados.
\pause

\item ¿Implementación en cliente o servidor?

\end{itemize}

\end{frame}

%-------------------------------------------------

\begin{frame}
\frametitle{Fase Online}

\begin{itemize}

\item Para la fase online se reconocen dos etapas principales.

\begin{enumerate}
\pause
\item Colectar un vector de señales RSSI en la posición actual del usuario, es decir, el vector de intensidad de la señal.
\pause
\item Proveer este vector de entrada a los algoritmos de aprendizaje supervisado
\end{enumerate}

\pause
\item Una vez que los algoritmos de clasificación proveen el resultado de la posición física, entonces la misma aplicación de la fase offline, es utilizada para mostrar en un mapa de tiempo real la localización actual de usuario.

\pause

\item Para realizar esta tarea se deben tener en cuenta las normalizaciones realizadas y aplicar correctamente la transformación PCA.


\end{itemize}
\end{frame}


%-------------------------------------------------

\begin{frame}
\frametitle{Proceso de desarrollo}

\begin{figure}
\includegraphics[width=0.7\textwidth]{../figures/propuesta_memoria.png}
\end{figure}


\end{frame}

%-------------------------------------------------
\section{Experimentación}
%-------------------------------------------------

\subsection{Implementación}

\begin{frame}
\frametitle{Beacons y configuración}

\begin{columns}[t] % The "c" option specifies centered vertical alignment while the "t" option is used for top vertical alignment

\column{.5\textwidth} % Left column and width
\begin{figure}
\includegraphics[width=0.6\textwidth]{../figures/kontaktapp1.png}
\end{figure}

\column{.5\textwidth} % Right column and width
\begin{figure}
\includegraphics[width=0.6\textwidth]{../figures/kontaktapp2.png}
\end{figure}

\end{columns}
\end{frame}
%------------------------------------------------

\begin{frame}
\frametitle{Lugar de experimentación}

Estacionamiento subterráneo de la universidad Técnica Federico Santa María, Campus San Joaquín.

\begin{columns}[t] % The "c" option specifies centered vertical alignment while the "t" option is used for top vertical alignment

\column{.5\textwidth} % Left column and width
\begin{figure}
\includegraphics[width=\textwidth]{../figures/estSubterraneo.png}
\end{figure}

\column{.5\textwidth} % Right column and width
\begin{figure}
\includegraphics[width=\textwidth]{../figures/estReal.jpg}
\end{figure}

\end{columns}

\end{frame}

%------------------------------------------------

\begin{frame}
\frametitle{Software Utilizado}

\begin{itemize}
\item Aplicación Android:
	\begin{enumerate}[1]
\onslide<1->{\item Mostrar el plano del lugar de experimentación.}

\onslide<2->{\item Permitir la adición de nuevos dispositivos Beacons.}

\onslide<3->{\item Permitir la captura de datos, es decir, los nuevos Fingerprints. }

\onslide<4->{\item Modificar los valores de intervalo y el número de mediciones en cada punto, el cual puede también definirse en periodo de tiempo.}

\onslide<5->{\item Tener una base de datos \textit{SQLite}.}

\onslide<6->{\item Para la etapa online, debe permitir seleccionar el algoritmo a utilizar y mostrar en tiempo real la posición del usuario.}

\end{enumerate}

\onslide<7->{\item Scikit-learn}

\onslide<8->{\item Tensorflow}

\end{itemize}

\end{frame}

%------------------------------------------------

\begin{frame}
\frametitle{Recolección de Fingerprints}

\begin{columns}[t] % The "c" option specifies centered vertical alignment while the "t" option is used for top vertical alignment

\column{.7\textwidth} % Left column and width

\begin{itemize}
\onslide<1->{\item Samsung Galaxy J7 Prime,CPU Octa-core 1.6 GHz Cortex-A53, una GPU Mali-T830 MP1, 3GB de memoria ram interna y el tipo de Bluetooth corresponde a 4.1 LE}

\onslide<2->{\item 8 beacons en un área reducida del estacionamiento y ubicar cada Beacon a una distancia de 16 metros a sus vecinos adyacentes. 16 x 44 metros ($704m^2$). }


\onslide<3->{\item Grilla para los puntos de medición o referencia de 4 metros por 4 metros, 44 en total. }
\end{itemize}

\column{.5\textwidth} % Right column and width
\onslide<1->{\begin{figure}
\includegraphics[width=0.65\textwidth]{../figures/deployBeacons.png}
\end{figure}
}

\end{columns}

\end{frame}

%------------------------------------------------

\begin{frame}
\frametitle{Recolección de Fingerprints}

\begin{columns}[t] % The "c" option specifies centered vertical alignment while the "t" option is used for top vertical alignment

\column{.7\textwidth} % Left column and width

\begin{itemize}
\onslide<1->{\item 3000 mediciones por posición. Se decide inspeccionar y recolectar datos a través de diferentes días.}

\onslide<2->{\item Se seleccionan 150 mediciones por punto de la grilla, para tener menor información repetida y no sobre muestrear la base de datos.}


\onslide<3->{\item Se obtiene una base de datos \textit{SQLite} con Figerprints, la cual presenta 6600 registros}

\onslide<4->{\item Clases X e Y son las posiciones asociadas a cada coordenada.}
\end{itemize}

\column{.5\textwidth} % Right column and width
\onslide<1->{\begin{figure}
\includegraphics[width=\textwidth]{../figures/ejemplo_csv.png}
\end{figure}
}

\end{columns}

\end{frame}

%-------------------------------------------------
\begin{frame}
\frametitle{Entrenamiento de clasificadores}

\begin{columns}[t] 
\column{.5\textwidth} % Left column and width

\only<1>{\begin{figure}
\includegraphics[width=\textwidth]{../figures/NB.png}
\end{figure}}

\only<2>{\begin{figure}
\includegraphics[width=\textwidth]{../figures/SVM-Lineal.png}
\end{figure}}

\only<3>{\begin{figure}
\includegraphics[width=\textwidth]{../figures/decision-tree.png}
\end{figure}}

\only<4>{\begin{figure}
\includegraphics[width=\textwidth]{../figures/adaboost.png}
\end{figure}}

\column{.5\textwidth} % Right column and width
\only<1>{\begin{figure}
\includegraphics[width=\textwidth]{../figures/SVM-RBF.png}
\end{figure}}

\only<2>{\begin{figure}
\includegraphics[width=\textwidth]{../figures/knn-results.png}
\end{figure}}

\only<3>{\begin{figure}
\includegraphics[width=\textwidth]{../figures/random-forest.png}
\end{figure}}

\only<4>{\begin{figure}
\includegraphics[width=\textwidth]{../figures/qda.png}
\end{figure}}

\end{columns}


\end{frame}

%-------------------------------------------------
\begin{frame}
\frametitle{Entrenamiento de clasificadores}

\begin{itemize}
\item Red utilizada es una red neuronal profunda con dos capas ocultas, la primera de ellas tiene 256 neuronas o nodos, mientras que la segunda capa posee 64 neuronas.

\item 20000 epoch, \textit{learning rate}  igual a $\alpha = 0.3$ . También se define un \textit{batch size} igual a 32.
\end{itemize}
\begin{columns}[t] 
\column{.5\textwidth} % Left column and width

\begin{figure}
\includegraphics[width=\textwidth]{../figures/nn_estructura.png}
\end{figure}



\column{.5\textwidth} % Right column and width
\begin{figure}
\includegraphics[width=\textwidth]{../figures/nn_plot.png}
\end{figure}

\end{columns}


\end{frame}


%-------------------------------------------------
\begin{frame}
\frametitle{Tabla de entrenamiento}

\begin{table}[ht!]
\centering
\resizebox{\textwidth}{!}{%
\begin{tabular}{|c|c|c|c|c|}
\hline
Algoritmo                     & Accuracy & Error medio X & Error medio Y & Error Absoluto \\ \hline
NN                            & 97.94\%  & 0.1579        & 0.0735        & 0.1741         \\ \hline
$SVM(RBF, C=1, \gamma = 4)$   & 96.81\%  & 0.2254        & 0.1018        & 0.2473         \\ \hline
$KNN(k = 2)$                  & 95.43\%  & 0.9842        & 0.1575        & 0.9967         \\ \hline
QDA                           & 85.25\%  & 5.1103        & 4.7175        & 6.9548         \\ \hline
$SVM(Lineal, C=1)$            & 78.75\%  & 9.3163        & 6.0387        & 11.1022        \\ \hline
Random Forest                 & 77.33\%  & 11.3430       & 3.1409        & 11.7698        \\ \hline
Naive Bayes                   & 71.73\%  & 12.2303       & 9.4836        & 15.4763        \\ \hline
Decision Tree( max depth = 5) & 57.91\%  & 57.8012       & 5.6412        & 58.0758        \\ \hline
Adaboost                      & 26.03\%  & 150.8848      & 6.5333        & 151.0261       \\ \hline
\end{tabular}
}
\end{table}


\end{frame}

%-------------------------------------------------
\begin{frame}
\frametitle{Entrenamiento utilizando PCA}

\begin{itemize}
\item Número de componentes principales que deben ser utilizadas para disminuir los tiempos de procesamiento y el número de componentes no ortogonales, es decir, reducir la información redundante total.
\pause
\item No existe un algoritmo que lo determine automáticamente.
\pause
\item Se proponen tres métodos para encontrar las componentes principales:

\begin{enumerate}[1]
\pause
	\item El primer método se basa en la información contextual presente en los valores propios.
	\pause
	\item Para el segundo método, es necesario establecer la suma acumulada porcentual de la varianza explicada.
	\pause
	\item Con respecto al tercer método, este se basa en seleccionar las primeras componentes principales según los resultados obtenidos en los clasificadores.
\end{enumerate}
\end{itemize}

\end{frame}

%-------------------------------------------------
\begin{frame}
\frametitle{Entrenamiento utilizando PCA}

\only<1>{\begin{columns}[t] % The "c" option specifies centered vertical alignment while the "t" option is used for top vertical alignment

\column{.5\textwidth} % Left column and width

\begin{figure}
\includegraphics[width=\textwidth]{../figures/eigenvalues.png}
\end{figure}

\column{.5\textwidth} % Right column and width
\begin{figure}
\includegraphics[width=\textwidth]{../figures/varianza_ratio.png}
\end{figure}
 % Right column and widt

\end{columns}}

\only<2>{\begin{figure}
\includegraphics[width=\textwidth]{../figures/comparativa_clasificadores_pca.png}
\end{figure}}


\end{frame}

%-------------------------------------------------
\begin{frame}
\frametitle{Entrenamiento utilizando PCA}

\begin{table}[ht!]
\centering
\resizebox{\textwidth}{!}{%
\begin{tabular}{|c|c|c|c|c|}
\hline
Algoritmo                     & Accuracy & Error medio X & Error medio Y & Error Absoluto \\ \hline
NN                            & 93\%  & 1.8956        & 0.6589        & 2.0068         \\ \hline
$SVM(RBF, C=1, \gamma = 4)$   & 92.39\%  & 2.4581  & 0.7830        & 2.5797         \\ \hline
$KNN(k = 2)$                  & 90.83\%  & 2.1381 & 0.4872        & 2.1929         \\ \hline
QDA                           & 71.25\%  & 15.9418 & 9.3042        & 18.4583         \\ \hline
$SVM(Lineal, C=1)$            & 66.20\%  & 19.5878 & 10.4824        & 22.2162        \\ \hline
Random Forest                 & 64.15\%  & 21.3284 & 5.4472        & 22.0130       \\ \hline
Decision Tree( max depth = 5) & 56.96\%  & 33.5151       & 8.9163       & 34.6808        \\ \hline
Naive Bayes                   & 50.71\%  & 31.3406 & 10.0727        & 32.9194        \\ \hline
Adaboost                      & 32.20\%  & 73.9345 & 9.3042       & 74.5176       \\ \hline
\end{tabular}
}
\end{table}


\end{frame}


%-------------------------------------------------
\begin{frame}
\frametitle{Fase Online}

\begin{itemize}
\item No hay forma de determinar el error absoluto, producto de que para ello se debe proporcionar la posición real.

\pause
\item Lo que se propone para determinar los resultados son dos formas llamadas método estático y método dinámico.
\begin{enumerate}[1]
\pause
\item En el método estático lo que se hace es permanecer quieto en un determinado punto durante un tiempo predefinido. El tiempo utilizado en este caso corresponde a 15 minutos.

\pause
\item Para el caso del método dinámico, lo que se busca es abarcar la mayor cantidad de puntos posibles. En este caso se decide hacer una caminata a través de todos los 44 puntos.
\end{enumerate}
\end{itemize}


\end{frame}

%-------------------------------------------------
\begin{frame}
\frametitle{Fase Online}

\begin{columns}[t] % The "c" option specifies centered vertical alignment while the "t" option is used for top vertical alignment

\column{.3\textwidth} % Left column and width
\begin{figure}
\includegraphics[width=\textwidth]{../figures/fase_online1.png}
\end{figure}

\column{.3\textwidth} % Right column and width
\begin{figure}
\includegraphics[width=\textwidth]{../figures/fase_online2.png}
\end{figure}

\column{.3\textwidth} % Right column and width
\begin{figure}
\includegraphics[width=\textwidth]{../figures/fase_online3.png}
\end{figure}

\end{columns}

\end{frame}

%-------------------------------------------------
\section{Resultados}
%-------------------------------------------------

\subsection{Métricas Obtenidas}
\begin{frame}
\frametitle{Errores medios método dinámico}

Método dinámico sin PCA

\begin{table}[ht!]
\centering
\resizebox{\textwidth}{!}{%
\begin{tabular}{|c|c|c|c|c|c|}
\hline
Clasificador & Error x & Error y & Varianza x & Varianza y & RMSE    \\ \hline
KNN          & 1.5858  & 4.6391  & 7.1970     & 2.1780     & 6.9323  \\ \hline
SVM          & 6.8207  & 2.8874  & 1.7243     & 0.5989     & 10.0323 \\ \hline
NN           & 4.3784  & 3.9113  & 13.0950    & 5.0712     & 8.2994  \\ \hline
\end{tabular}
}
\end{table}

Método dinámico con PCA
\begin{table}[!ht]
\centering
\resizebox{\textwidth}{!}{%
\begin{tabular}{|c|c|c|c|c|c|}
\hline
Clasificador & Error x & Error y & Varianza x & Varianza y & RMSE   \\ \hline
KNN PCA      & 2.0023  & 4.3983  & 7.5113     & 2.0696     & 6.6812 \\ \hline
SVM PCA      & 6.8948  & 2.4257  & 4.1134     & 2.0348     & 9.5668 \\ \hline
NN PCA       & 5.8874  & 4.4513  & 8.6088     & 3.2089     & 9.5188 \\ \hline
\end{tabular}
}
\end{table}


\end{frame}

%-------------------------------------------------
\begin{frame}
\frametitle{Errores medios método estático}

Método estático sin PCA

\begin{table}[!h]
\centering
\resizebox{\textwidth}{!}{%
\begin{tabular}{|c|c|c|c|c|c|}
\hline
Clasificador & Error x & Error y & Varianza x & Varianza y & RMSE   \\ \hline
KNN          & 3.2385  & 1.5417  & 3.3321     & 1.0940     & 5.0520 \\ \hline
SVM          & 5.2493  & 1.6986  & 2.0598     & 0.7594     & 7.0241 \\ \hline
NN           & 3.7300  & 2.1937  & 5.8885     & 0.7373     & 4.4857 \\ \hline
\end{tabular}
}
\end{table}

Método estático con PCA
\begin{table}[!h]
\centering
\resizebox{\textwidth}{!}{%
\begin{tabular}{|c|c|c|c|c|c|}
\hline
Clasificador & Error x & Error y & Varianza x & Varianza y & RMSE   \\ \hline
KNN PCA      & 2.9892  & 1.4475  & 3.1487     & 1.6391     & 5.0340 \\ \hline
SVM PCA      & 3.2313  & 1.4905  & 3.0630     & 1.7817     & 5.1757 \\ \hline
NN PCA       & 1.5578  & 1.7488  & 3.8045    & 2.6885     & 3.9341 \\ \hline
\end{tabular}
}
\end{table}

\end{frame}
%-------------------------------------------------
\begin{frame}
\frametitle{Cumulative distribution function Dinámico}

\begin{equation}
F_{X} (x) = P(X \le x)
\end{equation}

\begin{columns}[t] % The "c" option specifies centered vertical alignment while the "t" option is used for top vertical alignment

\column{.5\textwidth} % Left column and width

\only<1>{\begin{figure}
\includegraphics[width=\textwidth]{../figures/cdf-knn-dinamico.png}
\end{figure}}

\only<2>{\begin{figure}
\includegraphics[width=\textwidth]{../figures/cdf-svm-dinamico.png}
\end{figure}}

\only<3>{\begin{figure}
\includegraphics[width=\textwidth]{../figures/cdf-nn-dinamico.png}
\end{figure}}

\column{.5\textwidth} % Right column and width
\only<1>{\begin{figure}
\includegraphics[width=\textwidth]{../figures/cdf-knnPCA-dinamico.png}
\end{figure}}

\only<2>{\begin{figure}
\includegraphics[width=\textwidth]{../figures/cdf-svmPCA-dinamico.png}
\end{figure}}

\only<3>{\begin{figure}
\includegraphics[width=\textwidth]{../figures/cdf-nnPCA-dinamico.png}
\end{figure}}

\end{columns}

\only<4>{\begin{table}[!h]
\centering
\resizebox{\textwidth}{!}{%
\begin{tabular}{|c|c|c|c|}
\hline
Clasificador & Sin PCA & Con PCA & Mejora     \\ \hline
KNN          & 16.576  & 16.1554 & 2.5374 \%  \\ \hline
SVM          & 20.8962 & 19.3874 & 7.2204 \%  \\ \hline
NN           & 20.5677 & 16.1554 & 21.4525 \% \\ \hline
\end{tabular}
}
\end{table}}

\end{frame}

%-------------------------------------------------
\begin{frame}
\frametitle{Cumulative distribution function Estático}

\begin{equation}
F_{X} (x) = P(X \le x)
\end{equation}

\begin{columns}[t] % The "c" option specifies centered vertical alignment while the "t" option is used for top vertical alignment

\column{.5\textwidth} % Left column and width

\only<1>{\begin{figure}
\includegraphics[width=\textwidth]{../figures/cdf-knn-estatico.png}
\end{figure}}

\only<2>{\begin{figure}
\includegraphics[width=\textwidth]{../figures/cdf-svm-estatico.png}
\end{figure}}

\only<3>{\begin{figure}
\includegraphics[width=\textwidth]{../figures/cdf-nn-estatico.png}
\end{figure}}

\column{.5\textwidth} % Right column and width
\only<1>{\begin{figure}
\includegraphics[width=\textwidth]{../figures/cdf-knnPCA-estatico.png}
\end{figure}}

\only<2>{\begin{figure}
\includegraphics[width=\textwidth]{../figures/cdf-svmPCA-estatico.png}
\end{figure}}

\only<3>{\begin{figure}
\includegraphics[width=\textwidth]{../figures/cdf-nnPCA-estatico.png}
\end{figure}}

\end{columns}

\only<4>{\begin{table}[!h]
\centering
\resizebox{\textwidth}{!}{%
\begin{tabular}{|c|c|c|c|}
\hline
Clasificador & Sin PCA  & Con PCA & Cambio    \\ \hline
KNN          & 16.1554  & 16.1554 & 0 \%      \\ \hline
SVM          & 16.1554  & 15.1327 & 6.33 \%   \\ \hline
NN           & 11.18033 & 16.1554 & -44.49 \% \\ \hline
\end{tabular}
}
\end{table}}

\end{frame}

%-------------------------------------------------
\begin{frame}
\frametitle{Análisis de distribución}

\begin{figure}
\includegraphics[width=\textwidth]{../figures/boxplot.png}
\end{figure}

\end{frame}

%-------------------------------------------------
\begin{frame}
\frametitle{Análisis tiempos de ejecución}

Resultados en términos de milisegundos con su respectiva mejora.

\begin{table}[ht!]
\centering
\resizebox{\textwidth}{!}{%
\begin{tabular}{|c|c|c|c|}
\hline
\textbf{Clasificador} & \textbf{Sin PCA} & \textbf{Con PCA} & \textbf{Incremento} \\ \hline
KNN                   & 64.9642          & 59.6786          & 8.1361\%            \\ \hline
SVM                   & 54.5985          & 25.6085          & 53.0966\%           \\ \hline
NN                    & 0.7610           & 0.5777           & 24.0867\%           \\ \hline
\end{tabular}
}
\end{table}

\end{frame}

%-------------------------------------------------
\begin{frame}
\frametitle{Análisis de resultados}

\begin{itemize}

\item Resultados estáticos son mucho mejores que los resultados dinámicos, sin embargo, el escenario de que el usuario este estático en un punto no es para nada realista.
\pause

\item Primero, los mejores valores de error medio son obtenidos por KNN y NN, en ambos métodos (estático y dinámico).
\pause

\item KNN es mucho menos disperso en ambos métodos y sus errores están más centrados en valores bajos, mientras NN presenta mucho mayor dispersión en el método dinámico, pero casi nada en el método estático, sobre todo al no utilizar PCA.
\pause

\item Mejor algoritmo es redes neuronales, a pesar de su distribución, mantiene valores bajos de error y tiempos de procesamiento.

\end{itemize}



\end{frame}

%-------------------------------------------------
\section{Conclusiones}
%-------------------------------------------------
\begin{frame}
\begin{columns}[t] % The "c" option specifies centered vertical alignment while the "t" option is used for top vertical alignment

\column{.5\textwidth} % Left column and width

$P_1$, $P_2$, $P_3$, $r_1$, $r_2$ y $r_3$ conocidos
¿Cuál es la posición de $B$?

\vspace*{.1\textwidth}

\begin{equation*}
x=\frac{r_1^2-r_2^2-d^2}{2d}
\end{equation*}
\begin{equation*}
y=\frac{r_1^2-r_3^2-x^2+i^2+j^2}{2j}-\frac{i}{j}x
\end{equation*}
\begin{equation*}
z=\pm\sqrt{r_1^2-x^2-y^2}
\end{equation*}

\column{.5\textwidth} % Right column and width
\begin{figure}
\includegraphics[width=\textwidth]{../figures_chesta/estado_del_arte/trilateration}
\end{figure}

\end{columns}

\end{frame}

%------------------------------------------------

\subsection{Tecnologías que permiten la geolocalización}

\begin{frame}
\frametitle{Tecnologías que permiten la geolocalización}

\begin{columns}[t]

\column{.5\textwidth}
\textbf{Posicionamiento \textit{outdoor}}

\begin{itemize}
\item Sistemas satelitales (GPS, GLONASS, Galileo, Beidou)
\item Localización por antenas móviles (GSM)
\end{itemize}


\column{.5\textwidth}
\textbf{Posicionamiento \textit{indoor} (IPS)}

\begin{itemize}
\item Wi-Fi
\item Bluetooth
\item RFID
%\item Magnetismo
\end{itemize}

\end{columns}

\end{frame}

%------------------------------------------------

\begin{frame}
\frametitle{Posicionamiento \textit{outdoor}}

\begin{columns}[t]

\column{.5\textwidth}

\textbf{GPS}

\begin{itemize}
\item Red de 24 satélites 
\item Precisión del orden de centímetros a unos pocos metros
\item Requiere línea de visión directa (\textit{Line of Sight})
\end{itemize}

\column{.55\textwidth}

\textbf{GSM}

\begin{itemize}
\item Localización principalmente por Celdas de Origen y triangulación
\item Precisión del orden de 50m a 4km
\item Menor gasto energético
\end{itemize}



\end{columns}
\end{frame}

%------------------------------------------------

\begin{frame}
\frametitle{Posicionamiento \textit{indoor} - WiFi}

\begin{block}{Free-space path loss (FSPL)}
FSPL es la pérdida de la intensidad de señal que ocurre cuando una onda electromagnética viaja desde un transmisor a un receptor a través de una línea de visión directa en un espacio libre.	
\end{block}

\begin{figure}
\includegraphics[width=.65\textwidth]{../figures_chesta/estado_del_arte/fspl}
\end{figure}

\end{frame}

%------------------------------------------------
\begin{frame}
\frametitle{Posicionamiento \textit{indoor} - WiFi}

\begin{columns}

\column{.60\textwidth}

\begin{equation*}
FSPL=\Bigg(\frac{4\pi df}{c}\Bigg)^2
\end{equation*}

\begin{equation*}
FSPL(dB)=20 log(d)+20 log(f) + K
\end{equation*}

\begin{equation*}
d=10^{\frac{1}{20}(K-20 log(f)+FSPL)}
%d=10^{\frac{K-20 log(f)+FSPL}{20}}
\end{equation*}

\column{.5\textwidth}

\begin{figure}
\includegraphics[width=\textwidth]{../figures_chesta/estado_del_arte/fspl}
\end{figure}

\end{columns}

\end{frame}

%------------------------------------------------

\begin{frame}
\frametitle{Posicionamiento \textit{indoor} - Bluetooth}

\begin{itemize}
\item Bluetooth 4.0 (\textit{Bluetooth Low Energy})
\item Beacons
\end{itemize}

\begin{figure}
\includegraphics[width=.5\textwidth]{../figures_chesta/estado_del_arte/beacons_estimote}
\end{figure}

\end{frame}

%------------------------------------------------

\begin{frame}
\frametitle{Posicionamiento \textit{indoor} - Bluetooth}

\begin{block}{Tx Power}
Potencia constante transmitida por cada Beacon. A medida que la señal se aleja del beacon va decayendo su valor.
\end{block}

\begin{block}{RSSI}
Escala de referencia para medir el nivel de potencia de las señales recibidas por un dispositivo.  
\end{block}



\begin{equation*}
d = 0,899\Bigg(\frac{RSSI}{TxPower}\Bigg)^{7,771}+0,111
\end{equation*}



\end{frame}

%------------------------------------------------

\begin{frame}
\frametitle{Posicionamiento \textit{indoor} - RFID}

\begin{columns}[c]

\column{.5\textwidth}

\begin{itemize}

\item Posee tres componentes
\begin{enumerate}[1]
\item Lector de etiquetas
\item Ordenador central
\item Transpondedor
\end{enumerate}

\visible<2->{
\item Posicionamiento basado en celdas de origen}

\end{itemize}

 


\column{.5\textwidth}

\begin{figure}
\includegraphics[width=\textwidth]{../figures_chesta/estado_del_arte/how_rfid_works}
\end{figure}

\end{columns}


\end{frame}

%------------------------------------------------

\begin{comment}

\begin{frame}
\frametitle{Posicionamiento \textit{indoor} - Magnetismo}

\begin{itemize}
\item Cada pieza de metal dentro de un edificio interfiere de manera única con el campo magnético de la Tierra
\item No requiere la instalación de componentes físicos extras
\item Es necesario mapear el edificio completo
\end{itemize}


\end{frame}

\end{comment}

%------------------------------------------------
\section{Diseño del Estudio}
%------------------------------------------------
\subsection{Cualidades y costos de tecnologías}
\begin{frame}
\frametitle{Cualidades y costos de tecnologías - WiFi}

\vspace*{-1cm}
\begin{table}[H]
\centering
\label{tab:rango_wifi}
\resizebox{\textwidth}{!}{
\begin{tabular}{|c|c|c|c|c|}
\hline
\begin{tabular}[c]{@{}c@{}}\textbf{Protocolo} \\ \textbf{802.11}\end{tabular} & \begin{tabular}[c]{@{}c@{}}\textbf{Frecuencia}\\ \textbf{{[}GHz{]}}\end{tabular} & \begin{tabular}[c]{@{}c@{}}\textbf{Banda ancha}\\ \textbf{{[}MHz{]}}\end{tabular} & \begin{tabular}[c]{@{}c@{}}\textbf{Rango indoor} \\ \textbf{aproximado {[}m{]}}\end{tabular} & \begin{tabular}[c]{@{}c@{}}\textbf{Rango outdoor}\\\textbf{ aproximado {[}m{]}}\end{tabular} \\ \hline \hline
a                                                           & 3.7/ 5                                                         & 20                                                              & 35                                                                         & 120                                                                        \\ \hline
b                                                           & 2.4                                                            & 20                                                              & 35                                                                         & 140                                                                        \\ \hline
g                                                           & 2.4                                                            & 20                                                              & 50                                                                         & 140                                                                        \\ \hline
n                                                           & 2.4/5                                                          & 20 - 40                                                         & 70                                                                         & 250                                                                        \\ \hline
ac                                                          & 5                                                              & 20/40/80/160                                                    & 35                                                                         & -                                                                          \\ \hline
\end{tabular}
}
\end{table}


\begin{itemize}
\pause
\item Precio: CLP\$17.990 - CLP\$315.790
\pause
\item Consumo promedio mensual: 5,4[kWh]
\begin{itemize}
\pause
\item Costo energético mensual: CLP\$607\only<4->{\footnote{Valor kWh: CLP\$112,36. Fuente: Enel}}
\end{itemize}
\end{itemize}

\end{frame}


%------------------------------------------------

\begin{frame}
\frametitle{Cualidades y costos de tecnologías - Bluetooth}

\vspace*{-1cm}
\begin{table}
\centering
\label{tab:rango_beacons}
\resizebox{\textwidth}{!}{
\begin{tabular}{c||c|c|c|c|}
\cline{2-5}
                                                                                        & \includegraphics[scale=0.3]{../figures_chesta/diseno_del_exp/location_beacon.png}                                                                                                                  & \includegraphics[scale=0.3]{../figures_chesta/diseno_del_exp/proximity_beacon.png}                                                                                                                                                                                      & \includegraphics[scale=0.3]{../figures_chesta/diseno_del_exp/sticker_beacon.png}                                                                                                                                                                                                      & \includegraphics[scale=0.3]{../figures_chesta/diseno_del_exp/video_beacon.png}                                                                                                                                                                                                                \\
                                                                                        & \textbf{Locación}                                                                                                          & \textbf{Proximidad}                                                           & \textbf{Sticker}                                                                             & \textbf{Video}                                                                                        \\ \hline \hline
\multicolumn{1}{|c||}{\begin{tabular}[c]{@{}c@{}}\textbf{Vida útil}\\ \textbf{batería}\end{tabular}}       & Hasta 5 años                                                                                                            & Hasta 2 años                                                               & Hasta 1 año                                                                               & \begin{tabular}[c]{@{}c@{}}- \\ (conectado por USB)\end{tabular}                             \\ \hline
\multicolumn{1}{|c||}{\textbf{Rango}}                                                             & Hasta 200 metros                                                                                                        & Hasta 70 metros                                                            & Hasta 7 metros                                                                            & Hasta 10 metros                                                                                    \\ \hline
\multicolumn{1}{|c||}{\textbf{Grosor}}                                                            & 24 mm                                                                                                             & 17 mm                                                                & 6 mm                                                                                & 14 mm                                                                                        \\ \hline
\multicolumn{1}{|c||}{\begin{tabular}[c]{@{}c@{}}\textbf{Dispositivos}\\ \textbf{en el kit}\end{tabular}}  & 3 beacons                                                                                                         & 3 beacons                                                            & 10 stickers                                                                         & 3 mirrors                                                                                    \\ \hline
\multicolumn{1}{|c||}{\textbf{Precio}}                                                            & USD\$99                                                                                                           & USD\$59                                                              & USD\$99                                                                             & USD\$99                                                                                      \\ \hline
\end{tabular}
}
\end{table}

\begin{itemize}
\pause
\item \textit{Plug \& Play}
\pause
\item Baterías de litio 3[V] - 620[mAh]
\begin{itemize}
\pause
\item Costo: CLP\$5.000 - CLP\$6.000
\pause
\item Costo energético mensual: CLP\$250
\end{itemize}
\end{itemize}

\end{frame}

%------------------------------------------------

\begin{frame}
\frametitle{Cualidades y costos de tecnologías - RFID}

\vspace*{-.5cm}
\begin{table}[H]
\centering
\label{tab:rfid}
\resizebox{\textwidth}{!}{
\begin{tabular}{|c||c|c|c|}
\hline
\textbf{Tipo}                  & LF                                                                                          & HF                                                                                                & UHF                                                                                                        \\ \hline \hline
\textbf{Frecuencia}            & 125 kHz                                                                                     & 13.5 MHz                                                                                          & 915 MHz                                                                                                    \\ \hline
\textbf{Alcance}               & \textless 2.0 m                                                                             & \textless 1.0 m                                                                                   & \textgreater 3.0 m                                                                                         \\ \hline
\textbf{Aplicaciones}          & \begin{tabular}[c]{@{}c@{}}Identificación \\ de animales, \\ control de acceso\end{tabular} & \begin{tabular}[c]{@{}c@{}}Monedero, \\ Pasaporte, Tarjeta BIP, \\ control de acceso\end{tabular} & \begin{tabular}[c]{@{}c@{}}Logística, Retail, \\ Caja, Pallet, \\ Identificación de vehículos\end{tabular} \\ \hline

\end{tabular}
}
\end{table}

\begin{itemize}
\pause
\item Precio: Desde USD\$568.50\only<2->{\footnote{https://www.atlasfridstore.com/}}
\begin{itemize}
\item Reader: Desde USD\$450
\item Antena (9m): USD\$79
\item Cable conexión: USD\$39 (2m) - USD\$114 (10m)
\item Tag RFID Pasivo: USD\$0.50 - USD\$2

\end{itemize}

\pause
\item Consumo promedio mensual: 9[kWh]
\begin{itemize}
\pause
\item energético mensual: CLP\$1.011
\end{itemize}
\end{itemize}

\end{frame}

%------------------------------------------------

\begin{frame}
\frametitle{Cualidades y costos de tecnologías - Resumen}

\begin{table}[H]
\centering
\resizebox{\textwidth}{!}{
\begin{tabular}{|c|c|c|c|}
\hline
\textbf{Tecnología} & \begin{tabular}[c]{@{}c@{}}\textbf{Rango por} \\ \textbf{dispositivo}\end{tabular}& \textbf{Costo unitario}     & \begin{tabular}[c]{@{}c@{}}\textbf{Costo mensual} \\ \textbf{unitario}\end{tabular}\\ \hline 
\hline
Wi-Fi      & \begin{tabular}[c]{@{}c@{}}50 metros (802.11g) a \\ 70 metros (802.11n)\end{tabular} & Desde CLP\$17.990  & CLP\$607               \\ 
[3ex]\hline
Bluetooth  & 70-200 metros         & Desde CLP\$13.223\footnotemark[5]  & CLP\$250               \\ [3ex]\hline
RFID       & Desde 5 metros        & Desde CLP\$382.242\footnotemark[5] & CLP\$1.011             \\ [3ex]\hline
\end{tabular}
}
\end{table}

\footnotetext[5]{Dólar observado el 02/07/2017: CLP\$672,37. \\Fuente: Banco Central de Chile.}

\end{frame}

%------------------------------------------------
\subsection{Lugar del estudio}

\begin{frame}
\frametitle{Lugar del estudio}

\begin{figure}
\includegraphics[width=\textwidth]{../figures_chesta/diseno_del_exp/plano_est}
\end{figure}

Estacionamiento subterráneo del Campus San Joaquín - Universidad Técnica Federico Santa María

\end{frame}

%------------------------------------------------
\begin{comment}
\subsection{Diseño de la Aplicación}

\begin{frame}
\frametitle{Diseño de la Aplicación}

\begin{itemize}

\item Aplicación móvil en Android

\item Requerimientos:
\begin{enumerate}[1]
\item Mostrar el plano de la ubicación
\item Permitir al usuario colocar marcadores de dispositivos Beacon/Access Point
\item Calcular la posición del usuario
\item Permitir al usuario agregar un marcador de la ubicación real
\item Calcular la distancia entre ubicación real y la calculada
\item Registrar las distancias en un archivo persistente
\end{enumerate}

\end{itemize}

\end{frame}
\end{comment}

%------------------------------------------------
\section{Implementación}
%------------------------------------------------

\subsection{Requerimientos}
\begin{frame}
\frametitle{Requerimientos}

\begin{enumerate}[1]
\pause
\item Mostrar el plano de la ubicación
\pause
\item Permitir al usuario colocar marcadores de dispositivos Beacon/Access Point
\pause
\item Calcular la posición del usuario
\pause
\item Permitir al usuario agregar un marcador de la ubicación real
\pause
\item Calcular la distancia entre ubicación real y la calculada
\pause
\item Registrar las distancias en un archivo persistente
\end{enumerate}


\end{frame}

%------------------------------------------------

\subsection{Ejecución}
\begin{frame}
\frametitle{Ejecución}

\begin{columns}

\column{.7\textwidth}

\begin{itemize}

\item Áreas de medición: \\7,95$[m^2]$ - 25,09$[m^2]$ - 27,64$[m^2]$ - 84,52$[m^2]$ - 118,37$[m^2]$
\visible<2->{
\item 200 mediciones por área}
\visible<3->{
\item Usuario inmóvil}
\visible<4->{
\item Método de mitigación: \textit{ventana deslizante}}

\end{itemize}


\column{.3\textwidth}

\begin{figure}
\includegraphics[width=\textwidth]{../figures_chesta/implementacion/triangle_area}
\end{figure}

\end{columns}

\end{frame}


%------------------------------------------------
\section{Resultados}
%------------------------------------------------

\begin{frame}
\frametitle{Área 7,95$[m^2]$}

\textbf{Posiciones calculadas}

\begin{figure}
\includegraphics[width=\textwidth]{../figures_chesta/resultados/posicion__7_95}
\end{figure}


\end{frame}

%------------------------------------------------

\begin{frame}
\frametitle{Área 7,95$[m^2]$}

\textbf{Errores entre posición real y calculada}

\begin{figure}
\includegraphics[width=\textwidth]{../figures_chesta/resultados/area__7_95}
\end{figure}



\end{frame}

%------------------------------------------------

\begin{frame}
\frametitle{Área 25,09$[m^2]$}

\textbf{Posiciones calculadas}

\begin{figure}
\includegraphics[width=\textwidth]{../figures_chesta/resultados/posicion__25_09}
\end{figure}


\end{frame}

%------------------------------------------------

\begin{frame}
\frametitle{Área 25,09$[m^2]$}

\textbf{Errores entre posición real y calculada}

\begin{figure}
\includegraphics[width=\textwidth]{../figures_chesta/resultados/area__25_09}
\end{figure}



\end{frame}

%------------------------------------------------

\begin{frame}
\frametitle{Área 27,64$[m^2]$}

\textbf{Posiciones calculadas}

\begin{figure}
\includegraphics[width=\textwidth]{../figures_chesta/resultados/posicion__27_64}
\end{figure}


\end{frame}

%------------------------------------------------

\begin{frame}
\frametitle{Área 27,64$[m^2]$}

\textbf{Errores entre posición real y calculada}

\begin{figure}
\includegraphics[width=\textwidth]{../figures_chesta/resultados/area__27_64}
\end{figure}



\end{frame}

%------------------------------------------------

\begin{frame}
\frametitle{Área 84,52$[m^2]$}

\textbf{Posiciones calculadas}

\begin{figure}
\includegraphics[width=\textwidth]{../figures_chesta/resultados/posicion__84_52}
\end{figure}


\end{frame}

%------------------------------------------------

\begin{frame}
\frametitle{Área 84,52$[m^2]$}

\textbf{Errores entre posición real y calculada}

\begin{figure}
\includegraphics[width=\textwidth]{../figures_chesta/resultados/area__84_52}
\end{figure}



\end{frame}

%------------------------------------------------

\begin{frame}
\frametitle{Área 118,37$[m^2]$}

\textbf{Posiciones calculadas}

\begin{figure}
\includegraphics[width=\textwidth]{../figures_chesta/resultados/posicion__118_37}
\end{figure}


\end{frame}

%------------------------------------------------

\begin{frame}
\frametitle{Área 118,37$[m^2]$}

\textbf{Errores entre posición real y calculada}

\begin{figure}
\includegraphics[width=\textwidth]{../figures_chesta/resultados/area__118_37}
\end{figure}



\end{frame}

%------------------------------------------------

\begin{frame}
\frametitle{Resumen resultados}

\begin{figure}
\includegraphics[width=\textwidth]{../figures_chesta/resultados/summary}
\end{figure}


%\begin{table}[H]
%\centering
%\label{tab:resumen_resultados}
%\resizebox{53.5ex}{!}{
%\begin{tabular}{ccccccc}
%\hline
%\textbf{Área {[}$m^2${]}}         & \multicolumn{2}{c}{7,95} & \multicolumn{2}{c}{25,09} & \multicolumn{2}{c}{27,64} \\ \hline
%\textbf{Tecnología}               & Bluetooth     & WiFi      & Bluetooth      & WiFi      & Bluetooth      & WiFi\\ \hline
%\textbf{Promedio [m]}      & 8,12          & 14,83     & 12,42          & 24,31     & 12,71          & 23,14     \\ \hline
%\textbf{Desv. estándar [m]} & 7,20          & 8,86      & 8,39           & 17,57     & 9,73           & 12,83\\ \hline
%\end{tabular}
%}
%\end{table}

%\begin{table}[H]
%\centering
%\label{tab:resumen_resultados}
%\resizebox{43ex}{!}{
%\begin{tabular}{ccccc}
%\hline
%\textbf{Área {[}$m^2${]} }        & \multicolumn{2}{c}{84,52} & \multicolumn{2}{c}{118,37} \\  \hline
%\textbf{Tecnología}               &  Bluetooth      & WiFi      & Bluetooth      & WiFi       \\ \hline
%\textbf{Promedio [m]}     &  30,89          & 25,86     & 33,02          & 26,22      \\ \hline
%\textbf{Desv. estándar [m]} &  23,21          & 15,45     & 22,90          & 17,77      \\ %\hline
%\end{tabular}
%}
%\end{table}

\end{frame}

%------------------------------------------------
\section{Conclusiones}
%------------------------------------------------

\begin{frame}
\frametitle{Conclusiones}

\begin{itemize}
\item Para áreas reducidas, Bluetooth es más efectivo que WiFi
\pause
\item Para áreas mayores, WiFi presenta un error más estable
\pause
\item La precisión y exactitud del posicionamiento depende de la densidad de dispositivos
\pause
\item Importancia en algoritmos de localización
\pause
%\item Numerosa presencia de señales inalámbricas
%\pause
\item El posicionamiento indoor aún es un campo abierto de estudio

\end{itemize}

\end{frame}

%------------------------------------------------

\begin{frame}
\Huge{\centerline{Gracias por su atención}}
\end{frame}

%------------------------------------------------

\end{document}
