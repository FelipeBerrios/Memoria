%!TEX root = ../memoria.tex

Lograr un posicionamiento exacto y con bajo error en interiores se ha convertido en una de las problemáticas más importantes del último tiempo, ya que otras tecnologías como GPS no se comportan de la forma esperada en este tipo de recintos, sobre todo en donde las señales les cuesta penetrar como edificios o lugares bajo tierra. Entonces, en esta memoria se presenta una forma de abarcar el problema del posicionamiento \textit{indoor}, mediante las redes inalámbricas del protocolo Bluetooth con las cuales utilizando su intensidad de señal y en conjunto con algoritmos de máquinas de aprendizaje y la técnica \textit{Fingerprint}, se plantea una solución para lograr posicionamiento en interiores. Para ello se realiza la experimentación en el estacionamiento subterráneo de la universidad Federico Santa María y se desarrolla una aplicación móvil Android para medir y probar los algoritmos. Los resultados muestran que el mejor algoritmo es redes neuronales con un error medio de 3.93 metros para el caso estático (sin movimiento) y k nearest neighbour con 6.68 metros en el caso dinámico (en movimiento). Además, se analizan las ventajas de usar técnicas de reducción de dimensionalidad como PCA, en donde la mayoría de los algoritmos se ven favorecidos, a excepción de redes neuronales. Finalmente se determina que redes neuronales es el mejor algoritmo, a pesar de su distribución dispersa, ya que presenta errores bajos y poco tiempo de procesamiento.

\vspace{20mm}

\paragraph{Palabras Clave.}
\LaTeX{}, Plantilla para Memoria, Departamento de Industrias, UTFSM.

