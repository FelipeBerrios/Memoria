%!TEX root = ../memoria.tex

Plantilla \LaTeX{} para las Memorias y Tesis del Departamento de Industrias, UTFSM.

Se incluyen también algunos ejemplos de cómo incorporar tablas y gráficos en distintas presentaciones respetando las Normas de Biblioteca para Memorias y Tesis de la UTFSM.

\vspace{20mm}

\paragraph{Palabras Clave.}
\LaTeX{}, Plantilla para Memoria, Departamento de Industrias, UTFSM.

\vspace{10mm}

\begin{framed}
\noindent\textbf{\color{red}Para el impaciente ...}

Abra el archivo de configuración \inlinecode{config.tex} para cambiar título, autor, fecha, etc. de la portada y del documento en general.

Abra  y compile el documento maestro \inlinecode{memoria.tex}. Si hay errores, verifique primero que todos los paquetes \LaTeX{} han sido instalados.

Si desea omitir alguna sección (dedicatoria, agradecimientos, etc.), revise el documento maestro \inlinecode{memoria.tex} y agregue o comente (o elimine) las líneas correspondientes.

Por ejemplo, para eliminar esta sección, borre las líneas:

\begin{Verbatim}[frame=lines, label=\inlinecode{memoria.tex} (extracto)
, fontsize=\footnotesize
, baselinestretch=1
, formatcom=\color{gray}]
\section*{RESUMEN EJECUTIVO}
\insertFile[plain]{resumen}}
\end{Verbatim}
\end{framed}

\newpage
\subsection*{¡Importante! [LEAME]}

\subsubsection*{Impresión por ambos lados.}
Este documento está preparado para ser impreso por ambos lados de una hoja (\emph{``twoside''}). Para cambiar esto, en la ``clase de documento'' (archivo \inlinecode{memoria.tex}), reemplazar la palabra \emph{``twoside''} por \emph{``oneside''}. Es por esto que encontrará algunas hojas que están en blanco, aparentemente sin motivo.


\begin{Verbatim}[frame=lines, label=\inlinecode{memoria.tex} (extracto)
, fontsize=\footnotesize
, baselinestretch=1
, formatcom=\color{gray}]
%---------------------------------------------------------------------------
%%% DOCUMENT CLASS
\documentclass[
    11pt,
    letterpaper,
    twoside
]{thesis_utfsm}
%---------------------------------------------------------------------------
\end{Verbatim}


Es posible que deba cambiar otras configuraciones también para imprimir por un sólo lado. En particular aquellas páginas en blanco después de los agradecimientos y dedicatoria.

Contribuye con el ahorro de papel, no ocupes más hojas de las necesarias.

\subsubsection*{Codificación de caracteres.}

Todos los archivos \inlinecode{*.tex} de esta plantilla han sido preparados ocupando la codificación de caracteres por defecto \emph{unicode} (UTF-8). Windows (y algunas versiones de OSX) ocupan otro tipo de codificación (ej. \emph{Windows-1252} o \emph{Mac Roman}).

Si deseas ocupar esta plantilla y encuentras problemas con los caracteres acentuados, entonces puedes optar por una de estas tres alternativas:
\begin{enumerate}[i)]
    \item cambiar tu editor (TexMaker, TexStudio, TexShop, etc.) para que ocupe UTF-8 como codificación de caracteres por defecto; o
    
    \item cambiar la codificación de cada documento \inlinecode{*.tex} para que ocupe la codificación nativa de tu sistema operativo; y, modificar el archivo \inlinecode{config.tex} la línea que dice:
    
    \inlinecode{\\usepackage[utf8x]\{inputenc\}}
    
    \item escribir todo ocupando caracteres pre-acentuados (ej. \inlinecode{\\'a} en lugar de á).
\end{enumerate}

\vspace{10mm}
\begin{framed}
    \noindent\textbf{Recuerde:} 
    
    Mezclar documentos de distintas codificaciones puede generarte muchos problemas al momento de compilar.  
\end{framed}

