%!TEX root = ../memoria.tex

\chapter{Definición del problema}

\section{Definición del problema}

En la actualidad, el sistema de posicionamiento global, GPS por sus siglas en inglés, se ha vuelto la tecnología de facto para el posicionamiento en exteriores, ya que permite una exactitud que puede alcanzar los 2 metros. El mayor problema radica en que GPS se basa en ondas con una frecuencia que habitualmente no traspasa objetos sólidos, con lo cual las señales enviadas por los múltiples satélites no traspasan algunas barreras con facilidad.

Cuando se usa tecnología GPS dentro de edificios o bajo tierra, existen muchos obstáculos e interferencia que empeoran significativamente la señal y exactitud, o simplemente las señales no pueden ser alcanzadas en el dispositivo. Con una frecuencia de 1575.42 MHz, GPS está calificado como una señal de alta frecuencia, por lo que su longitud de onda es corta y no puede atravesar objetos macizos. 

Para realizar la aproximación de posición de un determinado dispositivo o terminal, GPS utiliza 24 satélites a unos 20.200 kilómetros de altura. Cuando un determinado dispositivo desea saber su ubicación, debe contar con al menos tres satélites en su rango de visión, y utilizando la trilateración, se obtiene una aproximación de unos pocos metros hasta inclusive unos pocos centímetros.

Entre las principales fuentes de error en la calidad de los datos de posición que afectan a GPS, se encuentran principalmente los errores propios del satélite, como por ejemplo su reloj interno, errores orbitales y errores de la configuración geométrica entre los satélites. Por otro lado, existen errores a la recepción del dispositivo, como el ruido asociado a los datos o el centro de fase de antena. Finalmente, y más relativo a este trabajo, están los llamados errores de propagación, que son asociados al medio en donde se transportan las ondas, afectando significativamente los resultados obtenidos.

Para solucionar este problema, existen los denominados \textit{indoor positioning systems} (IPS), los cuales consisten en sistemas basados en redes inalámbricas, campos magnéticos, señales acústicas y otros métodos que utilizan los sensores internos de los teléfonos celulares. Algunos ejemplos de los problemas que resuelven estos sistemas son por ejemplo ayudar a encontrar tiendas dentro de un centro comercial, guiar personas discapacitadas visualmente en instalaciones o edificios, ayudar al transporte dentro de minas subterráneas para mover materiales. También se ha planteado sistemas de rescate en donde ante una eventualidad como por ejemplo un incendio, alguien alarma al sistema y este busca la salida más cercana dentro del edificio guiando al usuario \citep{5647401}.

Las tecnologías inalámbricas se han convertido en las más prometedoras dentro de los IPS, las cuales destacan debido al auge de los smartphones y la facilidad de utilizar redes como WiFi y Bluetooth. Muchas implementaciones de IPS se han realizado para WiFi, ya que este presenta un rango mucho más grande que Bluetooth, de hasta 35 metros, sin embargo, las señales en ocasiones pueden ser muy ruidosas. Bluetooth por su parte tiene un alcance en interiores de hasta 10 metros, pero con una precisión de hasta 1 metro.  Además, la última versión de Bluetooth denominada \textit{Bluetooth Low Energy} consume significativamente menos energía que WiFi.

Las técnicas actuales de posicionamiento \textit{indoor} cuentan con problemas como la reflexión en múltiples objetos como paredes, muebles o el mismo cuerpo humano. Este problema se denomina  “multi-path propagation”, y es uno de los principales causantes de error en la localización en interiores. Los métodos que confían plenamente en el indicador RSSI, el cual mide la fuerza de la señal, están sujetos a errores inherentes a la variación de las señales y al ruido.  Múltiples modelos matemáticos se utilizan para sobrellevar esta situación, entonces evaluar algoritmos que aprendan de la intensidad de las señales en cierta área es necesario para estimar de mejor manera la posición. Por lo tanto, el problema a abordar es mejorar la exactitud del posicionamiento \textit{indoor}, para lo cual se utiliza técnicas de aprendizaje automático.

\section{Objetivos}

El principal objetivo de este trabajo es determinar los algoritmos de aprendizaje automático que presenten mejores resultados en términos de precisión y exactitud para posicionamiento \textit{indoor} o en interiores, a partir de la experimentación en un recinto definido bajo ciertas condiciones experimentales.

\subsection{Objetivos Específicos}

\begin{itemize}
\item Evaluar calidad de las señales \textit{Bluetooth Low Energy}, tanto en precisión como exactitud.

\item Diseñar un método de mapeo para un área mediante señales RSSI (\textit{fingerprint}).

\item Comparar métodos de aprendizaje automático sobre mediciones RSSI para determinar cúal posee menor error y es más exacto.

\item Identificar límites del modelo propuesto, tanto es precisión, exactitud, error, distancia de alcance las señales, y otros factores variables.

\item Determinar que tanto afectan los métodos de reducción de dimensionalidad tanto en precisión, error y tiempo de procesamiento para los algoritmos de máquinas de aprendizaje estudiados.

\end{itemize}