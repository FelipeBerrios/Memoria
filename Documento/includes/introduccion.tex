%!TEX root = ../memoria.tex

\chapter{Introducción}

Al momento de definir la posición o localización, inmediatamente se manifiesta la idea de un sistema de referencia con sus respectivas coordenadas, es decir, un espacio de determinadas dimensiones, en el cual se puede medir y basado en un consenso común, establecer unidades para la correcta ubicación dentro de este.

La localización geográfica, es precisamente cualquier forma de localización dentro de un contexto geográfico. Desde la edad antigua, múltiples formas de localización han sido inventadas para ayudar a referenciar al ser humano, lo cual ha permitido facilitar así el comercio, la navegación y otros aspectos tan básicos y relevantes que hasta el día de hoy son requisitos para el correcto funcionamiento de la civilización.

Distintos tipos de indumentaria se utiliza para la navegación y localización, como son mapas, brújulas, relojes, telescopios, sextantes, entre otros. Dentro de los avances más importantes en este ámbito, es el desarrollo de la teoría científica y técnica denominada georreferenciación, la cual permite el posicionamiento espacial de una entidad en una localización geográfica y única, definida según un sistema de coordenadas y datum específicos. Tomando ventaja de lo anterior, es que se ha desarrollado el sistema de posicionamiento global o GPS, el cual mediante satélites y fusión de sensores puede georreferenciar cualquier dispositivo que presente un receptor GPS, sin importar el lugar del mundo en donde se encuentre y su conectividad. Gracias a GPS, el crecimiento y acceso de la georreferenciación y navegación está en progresivo aumento, a tal punto que cualquier persona con un smartphone puede saber su posición exacta, con un error de apenas unos pocos centímetros, con lo cual GPS es sin duda uno de los avances tecnológicos más grandes del último tiempo, ya que permite mejorar y optimizar rutas de comercio, flotas de transporte, ayuda además a ubicar direcciones, o también es fundamental en el uso militar o comercial.

Por todos los motivos anteriormente mencionado, GPS es la tecnología que lidera el posicionamiento en exteriores (\textit{outdoor}), sin embargo carece de la posibilidad de georreferenciar cuando las condiciones son adversas, por ejemplo, cuando la entidad a localizar está bajo tierra, o dentro de un edificio de múltiples pisos, ya que las ondas emitidas por los satélites no son capaces de penetrar las estructuras y se desvanecen, lo cual no permite el posicionamiento en interiores (\textit{indoor}). 


La localización en interiores es sin duda uno de los problemas más desafiantes del último tiempo, esto es debido a múltiples inconvenientes que han sido detectados por los experimentadores, por lo que localizar en interiores no sea tan sencillo como lo es en exteriores, lo cual tiene a muchas empresas e investigadores desarrollando soluciones según diversas perspectivas para así lograr mejorar la exactitud del posicionamiento en interiores, y con ello,  un estándar \textit{indoor} como lo es GPS en exteriores.

La presente memoria abarca este tema, el problema del posicionamiento \textit{indoor}, es decir, sin conectividad GPS en un lugar de experimentación en donde no se puede georreferenciar por los métodos comunes, lo cual establece la necesidad de buscar otras alternativas que presenten resultados favorables en este tipo de entornos y recintos.

La búsqueda de una solución en esta memoria nace a partir de la problemática de georreferenciar dentro de una explotación minera o mina, ya que, por el contexto, es difícil y costoso cablear para utilizar señales WiFi, por lo mismo, es necesario utilizar dispositivos más fáciles de instalar como son equipos Bluetooth y de bajo costo monetario. Además, dentro de la mina, las señales se ven sumamente afectadas por el entorno y la interferencia, con lo cual se requieren técnicas matemáticas que sean capaces de aprender patrones y disminuir este ruido.

La presente memoria entonces abarca el tema de posicionamiento indoor utilizando Bluetooth, combinándolo a su vez con técnicas de máquinas de aprendizaje, para determinar qué tan efectivo es el posicionamiento, además de su exactitud y error. La estructura de este trabajo se describe a continuación: El capítulo 2 define más específicamente el problema en cuestión, tratando así el contexto y la motivación con algunos casos reales. Además, se definen los objetivos base y alcance de la memoria. Luego en el capítulo 3, se establece el estado del arte, en donde se explican en extenso las formas más conocidas de abarcar el problema de localización en interiores, las tecnologías empleadas y técnicas matemáticas desarrolladas, cada una con sus ventajas y desventajas a modo de comparativa. El capítulo 4 describe la propuesta de solución desarrollada a lo largo del texto, además presenta una pequeña introducción a los dispositivos Beacons Bluetooth, algoritmos de máquinas de aprendizaje y técnicas de reducción de la dimensionalidad. El capítulo 5 presenta la experimentación, es decir, el lugar de experimentación, software y hardware utilizado, formato y tipo de pruebas realizado y construcción de la aplicación para posicionamiento. El capítulo 6 presenta los resultados obtenidos con un respectivo análisis, para el estudio del mejor algoritmo de posicionamiento. Finalmente, el capítulo 7 presenta las conclusiones de la presente memoria, con algunas formas de extender y mejorar los actuales sistemas de posicionamiento a modo de recomendación para futuras investigaciones.